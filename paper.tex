\documentclass[pageno]{jpaper}

\newcommand{\Semester}{Spring}
\newcommand{\IWreport}{2018}
\newcommand{\quotes}[1]{``#1''}
\makeatletter
\def\verbatim@font{\linespread{1}\normalfont\ttfamily}
\makeatother
\clubpenalty9998
\widowpenalty9999

\usepackage[normalem]{ulem}

\begin{document}

\title{
A Modular Framework for Developing, Deploying, and Evaluating Game-Theoretic Strategy Design Exercises}

\author{Andrew Wonnacott\\
  Advisers: Professor Dave Walker, Professor Matt Weinberg}

\date{}
\maketitle

\thispagestyle{empty}
\doublespacing{}
\begin{abstract}
  Student interest in Princeton's undergraduate course in algorithmic game theory, ``Economics and Computing'' (COS 445), has grown rapidly in recent years.
  This course employs small simulations to develop students' ability to use theoretical tools to analyze more real-world situations.
  These strategy design exercises require considerably more software infrastructure than would a usual theoretical problem set, including handout code for student self-evaluation and software autograders.
  While some frameworks for automated grading exist, none surveyed provided the specific features desired for repeatedly developing very similarly structured assignments and evaluating student submissions in the context of each other.
  In order to facilitate more automated grading of these assignments in this and future semesters, I have developed a framework of modular, extensible tools for building all of the necessary components of these strategy design assignments.
  Concurrently, in my role as a grader for COS 445, I used the framework to develop and evaluate four strategy design assignments, using assignment specifications developed with Professor Weinberg.
  This provides feedback from real student and instructor users which was used to improve the design.
  Empirical results demonstrate the importance of quality tooling to effective use of instructor time, low broken submission rates, and high student satisfaction.
\end{abstract}

\newpage
\section*{Introduction}
My work builds upon existing programming-based game theory exercises assigned in an existing Princeton course.
The intention of these strategy design exercises is to teach students how to use game theory to analyze a real world situation and work rationally, within healthy incentive structures where students do not try to hurt each others’ performance.
Each past offering of the course has rebuilt these assignments from scratch, and all associated instructor tooling, resulting in unnecessarily high allocation of instructor time to grading.
Additionally, the course has grown rapidly to become the single largest theoretical computer science course ever taught at Princeton.\footnote{registrar data here}\footnote{More students complete COS 340 in a given year, but it is taught in both fall and spring semesters.}
In order to eliminate unnecessary annual replication of previous coursework development, improve course performance at scale, and improve the course experience for the students, I develop an extensible framework to improve these strategy design exercises.

Four Princeton University departments offer courses in game theory: economics, politics, mathematics, and computer science.
While all of these courses teach the core concepts of game theory, each also connects this core to relevant examples and topics in its department.
For instance, in the computer science department, COS 445, ``Economics and Computing'', places emphasis on the connections between game-theoretic results and algorithms, software, and other aspects of computer science.
This is represented in the course topics, which focus on the relevance of game theory to matchings, auctions, cryptocurrencies, and other multi-actor computational systems.\footnote{https://www.cs.princeton.edu/~smattw/Teaching/cos445sp18.htm}
It is also represented in the design of the problem sets, which contain, in addition to traditional proof-based exercises, programming-based exercises which connect course topics to prerequesite computer science skills.
These programming-based exercises take a common form.

Students are presented with an iterated game, described both in the assignment handout and as a Java interface.
A classic iterated game is the iterated prisoner's dilemma\footnote{Figure: PD game -> Java iterface}, in which two players must collaborate to receive the best outcome, but are each incentivized to defect, which results in a less preferred outcome, and have acecss to the other player's play history when making their own decision.
While this game is well-studied and models some real-world incentive structures, the course also draws on various real-world phenomenon to design iterated games, including, from semester to semester, a college admissions game from the perspective of stable matchings, a search advertisement bidding game using auction theory, an election informed by voting theory, a gerrymandering-prevention simulation inspired by fair division, and exploitation of a fictional cryptocurrency by analyzing blockchain incentives.
These examples are not exhaustive, but illustrate the variety of course topics with potential for translation to iterated games.

Students then craft strategies to play the specified game, based on their own analyses of the game's equilbirium, and based on their own priors over the strategies which will be designed by their peers.
These strategies are evaluated on both on their design quality and on their empirical performance.
While design quality is evaluated on a purely individual basis, empirical performance is measured by simulating the play of the collection of submitted student strategies.
For instance, for the iterated prisoner's dilemma, every student strategy plays one full game (consisting of many iterations of the prisoner's dilemma) against each other strategy, and strategies are scored based off the average payoff they achieved.
For the search advertisement game, all the student strategies are concurrently asked to bid on the same search terms, and are scored based on their revenue from ad clicks (where each bidder knows the click-through rate for a term prior to bidding on that term) less their expenditures on advertising.
To create a more realistic simulation, students are graded solely the performance of their own strategy, not on a curve of the performance of their peers.
That is, student grades on these strategy design exercises are not zero-sum.

While the strategy design exercises assignments are popular with students\footnote{https://reg-captiva.princeton.edu/chart/index.php?terminfo=1184\&courseinfo=012095 requires CAS login}, they also demanded considerable instructor resources be dedicated to building and grading software resoruces, which have historically been significantly rebuilt for each assignment.
Worse still, due to course staffing changes, this work has not been preserved between semesters, resulting in inefficient allocation of department resources to producing new exercises without the institutional memory or software resources previously developed.
Due to the limited tooling provided in past semesters for students to evaluate their submissions, many submissions did not compile or caused runtime exceptions, requiring considerable additional instructor resources.

The goal of this project is to discover to what degree the development of new strategy design exercises can be simplified or automated, to improve the student resources available during the completion of these exercises, and to create institutional memory and decrease year-to-year repetition of the same work.
This was accomplished by designing a reuasble framework which embodies the commonality between strategy design exercises, allowing instructors creating new exercises to change only the necessary, assignment-specific details, and which will include more advanced automated grading tools than previously existed for the course.
Within this framework, I have built a collection of utilities to facilitate students' evaluation of their own strategies and to decrease strategy iteration overhead.

Through this work, I aim to improve the quality of computer science theory education, from the perspectives of both the students and faculty associated with COS 445.
Quantitative metrics, including software failure rates and instructor time usage, and qualitiative metrics, including student and instructor satisfaction, both illustrate the improvements to the course and emphasize the importance of quality tooling to quality programming-based assignments.

\section*{Related Work}
This work was developed to replace previous software design exercises assigned to COS 445 students\footnote{including the author} in Spring 2017.\footnote{https://web.archive.org/web/20170916042726/https://www.cs.princeton.edu/~smattw/Teaching/cos445sp17.htm}
In one problem set, students developed and implemented strategies for three different strategy design exercises, related to three different classical games: Prisoner's Dilemma, Centipede, and Ultimatum.
In a later problem set, an exercise placed students in the role of bidders in a generalized second-price auction.\footnote{an auction of multiple items, one after another, where the winning bid pays the second highest bid. This is the auction format used by Google to sell sponsored search auctions.}

Both problem sets were implemented and graded by Cyril Zhang\footnote{https://github.com/cyrilzhang}, a graduate student TA for the course.
Cyril developed interfaces, dummy implementations, and grader software for each strategy design exercise.\footnote{https://github.com/PrincetonUniversity/cos445-hw/tree/s17 requires invite: contact author or department}
However, this existing implementation of grader software for strategy design assignments is not modular and has to be completely rewritten for each assignment, and was designed for grader use only.
The student interfaces were documented, but the grader software was not documented for future re-use.
Additionally, some of the scripts and notebooks used by Cyril were not preserved and were not accessible as resources.
Thus, while these resources served as inspiration for this work and provided some assignment-specific code to my work developing specific assignments, I did not build directly on Cyril's code.

Previous to Spring 2017, COS 445 was taught by Professor Mark Braverman in Fall 2012 and Spring 2014\footnote{https://precourser.io/course/1184012095 requires Princeton CAS authentication}.
Though no documentation or archives of the course homepage from these semesters is available online or in the Internet Archive, I recovered some resources from the course account on the ``Cycles'' cluster of the Princeton Computer Science department and from the source code respositories used by the course staff.
In both semesters, Yonatan Naamad\footnote{https://github.com/ynaamad/} was responsible for software development.

The only resources available from Fall 2012 were collected from the course Google Code repository.\footnote{https://code.google.com/archive/p/cos445-scripts/}
This repository contains incomplete Python skeleton code for a basic tournament.
This was uploaded to the current course repository,\footnote{https://github.com/PrincetonUniversity/cos445-hw/tree/f12 requires invite: contact author or department} but was not reused.

The Spring 2014 iteration of the course was developed on a private GitHub repository.\footnote{https://github.com/ynaamad/cos445}
Through a clone of this repository stored on Cycles, I discovered that the course had PHP and Java implementations of strategy design exercises for Prisoner's Dilemma and another auction setting.
The Java code consisted of student handouts of interfaces and some testing code, but the testing code was not designed to be automated for grading and was not modular: it tested manually-enterred strategies against each other and assignment-specific details (e.g.\ game structure and payoffs) were integrated within the simulation code.
As a design decision I dismissed the available PHP code, which was used for course grading, due to language unfamiliarity.
These resources were uploaded to the current course repository,\footnote{https://github.com/PrincetonUniversity/cos445-hw/tree/s14 requires invite: contact author or department} but were not reused, in part because they were located and examined only when I was given access to the course site in order to deploy my project.

Beyond COS 445 at Princeton, many universities have taught computer science courses in algorithmic game theory.
However, a survey of several universities' courses, including courses at Harvard\footnote{https://beta.blogs.harvard.edu/k108875/}, Stanford\footnote{https://theory.stanford.edu/~tim/364a.html and https://theory.stanford.edu/~tim/f16/f16.html}, Cornell\footnote{https://www.cs.cornell.edu/courses/cs6840/2014sp/}, U.\ Washington\footnote{https://courses.cs.washington.edu/courses/cse522/05sp/}, Yale\footnote{https://zoo.cs.yale.edu/classes/cs455/fall11/}, U.\ Penn\footnote{https://www.cis.upenn.edu/~aaroth/courses/agtS15.html}, Georgia Tech\footnote{https://web.archive.org/web/20131002121909/https://www.cc.gatech.edu/\%7Eninamf/LGO10/index.html}, Carnegie Mellon\footnote{https://www.cs.cmu.edu/~arielpro/15896s16/index.html}, and Duke\footnote{https://www2.cs.duke.edu/courses/spring16/compsci590.4/} identified only one other course with similarly structured assignments.

Harvard's Computer Science 136, ``Economics and Computation'',\footnote{https://beta.blogs.harvard.edu/k108875/assignments/} assigns two programming problem sets.
The first requires implementation of a peer-to-peer filesharing system similar to Bittorrent, a protocol in which incentives parallel Prisoner's Dilemma.
However, the assignment is not a competition and student solutions are not differentiated based on strategic properties.
The second assignment is a strategy design exercise built on the generalized second-price auction, in which student strategies vary and are graded based on their performance in a competitive environment.
This is very similar to both the Spring 2014 and Spring 2017 auction exercises of COS 445 at Princeton.
However, the grading structure of this assignment is quite different from the grading structure of the COS 445 strategy design exercises, and the implementation is in Python.
While comparison between COS 445 assignments and CS 136 assignments in a future project might provide some pedagogical value, such work would be out of scope for this project.
Given that these assignments are implemented in Python, only handout (and not grading) code is available, and the assignments are graded differently, these resources were dismissed as a base for development of new resources for COS 445 at Princeton.

Many universities, including Princeton, maintain extensive systems for the automatic submission and processing of student assignments.
Such resources are useful in the general case of developing new programming-based assignments, but no such resources replicate the behavior implemented in this project.
Most autograding systems are designed with an assumption that student submission are graded in a sandbox, isolated from other student submissions.
Such systems are not applicable to this project, since the strategy design exercises involve evaluation of student strategy performance in a common environment.
However, existing resources for assignment submission and student feedback in use by Princeton are used in this assignment.
I integrated some CS DropBox\footnote{https://csguide.cs.princeton.edu/academic/csdropbox} functionality where relevant, and modelled the leaderboard structure after review of the leaderboard implementation used in other Princeton CS classes.
Generally, however, no such tools would leverage the high degree of structural similarity between strategy design exercises, and a dedicated project for these exercises will allow for more efficient development and evaluation.

\section*{Approach}
Cyril Zhang completed the TA requirement for the PhD program last fall, so at the start of this semester it was unclear who among the graduate and undergraduate course staff would be responsible for maintaining these assignments.
Thus, the course needed not just long-term work to improve student and instructor resources, but also short-term work to maintain and debug the same resources.
These harmonizing requirement suggest that the product should consist of one extensible implementation of all the reusable components for software design exercises, developed iteratively in conjunction with its usage in the course.
This software would naturally grow throughout development as demanded by the needs of the course, and the final product will naturally lower-bound the maximum degree of automation and modularity possible.

Before the first problem set of the course was assigned in this semester, Matt and I designed and I implemented a handout interface and tools to allow students to test their strategies against each other.
The game simulates college admissions using stable matchings.
While I implemented this handout with the explicit goal of producing a tool to test out multiple different student strategies in the specified application environment, I followed the principles of modular software design covered in previous courses.\footnote{Kernighan and Pike chapter 4}
I factored out the code which implemented the Gale-Shapley deferred acceptance algorithm for stable matchings from the code which amortized results from multiple randomized trials, and from agglomerated these results to output performance statistics for students.
Thus, when it came time to implement the second strategy design exercise, I was able to reuse code which processed results of trials, even though the nature of the trails changed from college admissions to elections.
Additionally, when it came time to grade the problem set, I was able to modify only the post-processing step to compute grades.
Thus, creating a new assignment requires constant work in the number of framework extensions, and creating a new framework extension requires constant work in the number of assignments.

The result of this process duplicates significant functionality from last year's assignments.
However, because of conscious design decisions to factor our repeated code from assignment implementations, it accomplishes what previous implementations had not.
It will be modular, as described above, so that new extensions such as additional tools or interfaces can be written for strategy design exercises.
Throughout the semester, I can leverage my role as an undergraduate course assistant for COS 445 during the current semester in order to identify and produce such extensions.
And, it will be reusable, both for the same assignments in future years, and for new strategy design exercises implemented by future course staff with lower overhead.

In order to increase the reusability, I elected to store it in a University-owned private GitHub repository.
This ensures that, even after I have graduated, future course staff will have access to the project.
Additionally, I documented the repository structure and each program's role in the repository, as documented later in this document.

\section*{Implementation}
The first key decision in the implementation of these programming assignments was language selection.
Many students suggested the assignments be implemented in Python,\footnote{per survey 1 results}, and this would be preferred by the course staff as well.
Members of the Mechanism Design group are not expected to have working proficiency with Java, but many have enough experience with Python to support students.
Additionally, Python's conciseness and ability to load classes from user input would be useful when implementing tools which are parametrized on user-input strategy names.
Unfortunately, COS 445 does not have Python experience as a course prerequisite, only Java experience.
Additionally, survey data from other course surveys suggests that 90\% of students report themselves fluent in Java after the Princeton introductory sequence, compared to 30\% for Python.\footnote{COS 333 survey data}

Thus, the whole system under consideration is constrained: it must support student strategy implementations in Java.
Given this constraint, and lacking any other language selection pressures, I elected to implement as much of the assignments as possible in Java, with limited usage of scripting languages such as make, Python, and bash.
These scripting languages were used where Java would not suffice and only for tasks for which implementation details were irrelevant to student assignment comprehension.

Java also offers convenience to assignment designers by enforcing an object-oriented design pattern.
This allows instructors to communicate an assignment specification formally through a well-documented interface.
Furthermore, by constructing new instances of the student implementations between game rounds, instructors can programatically enforce that students do not employ learning strategies outside the parameters of the assignment.\footnote{This requires that instructors forbid the usage of static variables, but it is trivial to verify compliance with this requirement.}

However, several default Java behaviors do not map cleanly onto the desired properties of strategy design exercises.
For instance, Java objects used as function parameters are passed by mutable, nullable reference.
This is in contrast to functional languages (including Ocaml and Rust), where parameters are immutable and not null unless otherwise explicitly specified.
Current industry standards for Java code obviate both of these issues using modern language features.\footnote{https://github.com/google/guava/wiki/UsingAndAvoidingNullExplained}\footnote{https://github.com/google/guava/wiki/ImmutableCollectionsExplained}\footnote{https://docs.oracle.com/javase/7/docs/api/java/util/Collections.html#unmodifiableList(java.util.List)}
In order to preserve backwards compatibility, these features are not implemented by default.
Where possible, I have taken advantage of these features.
Since the assignment specification documents that students are required not to modify parameters, instructor code can use unmodifiable (immutable) containers.
However, I have not included NotNull annotations because they are not taught in the prerequesite courses and, as previously noted, these assignments should not require students to self-teach any programming skills.
In my own framework code, which must be supported by future instructors, I decided to assume support for Java 9, released in fall 2016, and to use modern language features, since course instructors are at least as likely to have industry experience with modern Java code as experience with the Princeton introductory sequence.\footnote{https://www.techworld.com/news/developers/java-9-release-date-new-features-3660988/}

In order to design my framework, I reviewed unmodularized code from a past strategy design exercise, the Spring 2017 auctions exercise.
This is the only assignment to which I had access to the full implementation of the student handout (having taken the course) and the grader code (preserved by Professor Weinberg).
The first piece of the assignment to be implemented, after the game design is complete, is the student interface:\footnote{TODO: make this code a figure}
\begin{verbatim}
import java.util.List;

public interface Bidder {
    // Return your bid for the current day
    // Called once per day before the auction
    public double getBid(double dailyValue);

    // Let you know if you won, and how much the winners paid
    // Called once per day after the auction
    public void addResults(List<Double> bids, int myBid, double myPayment);
}
\end{verbatim}
This interface describes what students' strategies must do: they must, each day, bid on an auction of advertising slots which appear on a search results page (often called ``sponsored search results''), and they value the word being auctioned each day at a different value.
The strategy also has access the winning bids for each of the ten sponsored results from each past day, and knows if it won any auctions itself, and if so how much it paid.
This allows the strategy to obey the budget constraint: across all 1000 days, it may not spend more than 500.
These constants are not documented in the interface as designed, only in the assignment specification.
The instructors distribute a simple sample student implementation, in this case the truthful bidder.
Cyril also implemented a program used to grade the student-submitted strategies.
This program had two methods: \texttt{main} and \texttt{trial}.
The former ran 100 trials and printed out the results, averaging the outcomes across each trial for each student.
The latter contained an implementation of the generalized second-price auction format described in the assignment specification.

Having reviewed a former assignment, I implemented the first assignment handout for the course this semester: the college admissions simulation described previously.
In doing so, I made several implementation decisions in order to facilitate future reconfiguration of the game being played and of the postprocessing of the tournament results which are representative of design decisions made throughout the assignment.
In every assignment, the tournament simulation portion (instructor-written code) is broken up into game rounds, result processing and amortization, and further postprocessing, such as grade assignment and output formatting.
The divide between game rounds and code which processes game results was present in Cyril's auction code, as the separation between the \texttt{trial} and \texttt{main} methods.
However, this prior work did minimal amortization and no advanced output formatting, and had no separation between them.
While the handout code written for the first assignment did minimal postprocessing, this distinction was quite useful when later developing the autograder and additional extensions.

During the development of this handout, however, the statically-typed nature of Java raised some difficulty.
Cyril's code had student submission names hardcoded into complex expressions, as though they had been copied in from a manually-written list formatted using sed.
In the context of developing a handout to allow students to test multiple strategies of their own design, I determined requiring students to modify the simulator Java code whenever they changed which strategies they wanted to test would be so cumbersome that the simulator might not have been distributed at all.
Additionally, graders desire functionality to configure the list of strategies to be evaluated at grading time, without having to manually list out all the student strategy names, and to easily modify this list to identify and diagnose misbehaving student strategies.
Since Java is not able to type-check user-input strategy names, I determined that the most effective workaround for this issue would be to automatically rewrite the Java code before compile-time.

For this purpose, I developed a simple Python script capable of rewriting arbitrary template plaintext files using a CSV file.
This script, \texttt{formatBase.py}, builds a list of dictionaries, one for each row of the CSV, using the header of the CSV as the keys and the contents of the row as the values.
Then, it rewrites its input by finding lines which start with \texttt{//R} and end with \texttt{//}, and replaces each such line with, for each row of the CSV, the line rewritten by replacing each curly-enclosed key with the corresponding value from that row.
It also rewrites the key \texttt{0} with the row number from the configuration table.
This script directly injects the contents of the configuration file into the Java code with no escaping, as is necessary in this application.
However, this practice is extremely dangerous, since using a user-submitted configuration file would allow users to execute arbitrary code on the evaluation machine.
Fortunately, this is not a security concern, since students already run arbitrary code on the evaluation machine: their submitted strategy.

Given a CSV file as input with the header row \texttt{netID} and a row for each student's name, the following function is thus rewritten to instantiate one implementation of each Student class listed in the configuration file:
\footnote{use figure}
\begin{verbatim}
    public static int[] oneEachTrial(double S, double T, double W)
    {
        // Initialize students
        Student[] students = new Student[N];
        //R students[{0}] = new Student_{netID}();//
        return runTrial(students, S, T, W);
    }
\end{verbatim}
In the above code from the college applications simulator, \texttt{S}, \texttt{T}, and \texttt{W} are configuration parameters for the simulation which were documented in the student interface.
For contrast, the following method also instantiates an instance of an instructor-provided strategy using reflection.
This illustrates potential variation in data collection code as enabled by my modular design, and is used in the autograder described below.
\footnote{todo: figure}
\begin{verbatim}
    public static <Student_T extends Student> int[] withExtraTrial(
            Class<Student_T> clazz, double S, double T, double W)
    {
        // Initialize students
        Student[] students = new Student[N+1];
        //R students[{0}] = new Student_{netID}();//
        try {
            students[N] = clazz.getConstructor().newInstance();
        } catch (ReflectiveOperationException roe) {
            roe.printStackTrace();
            System.exit(1);
        }

        return runTrial(students, S, T, W);
    }
\end{verbatim}

now we can also have a better handout
outline better handout
* interface
* dumb implementation
* full copy of the above code
comment on use of formatbase here
comment on separate files for each assignment - why (simplify handout)

but we wanted a grader. how to build grader from that?
brief tangent to how to grade, not against peers
introduce replacement-comparison grading
* have some instructor strategies designed to perform well
* substitute, for each student, a copy of each instructor strategy
* grade based on how many instructor strategies the student's strategy did better than against the same pool
* comment about incentive structures of this method: correct due to amortization (maximize your expectation to maximize your chance of beating unknown instructor thresholds. amortization means your outcome equals your expectation)

issue with this: O(mn) time
* can cut down to O(m) time by just adding each instructor strategy one by one and seeing how they do against the whole field.
* assume that each individual student has negligible impact on that instructor's performance.
* Comment as to whether this is a realistic assumption.
* comment on incentives of this method - approximately correct as before

back to buidling the grader
manually write the list of grader strategies
make a new method very similar to one each trials: with extra trials
lookup table: how many instructor strategies beat -> score
didn't need to change the trial at all
so that doesn't need to be changed when makign a new assignment
we can now grade every assignment

catch: sometimes we want to simulate multiple parameter settings for student code
involves changing the main method a bit
not much change

so now we have a framework which makes it easy to build new assignments
what really changes?
build ultimatum

we also have a framework to build new tools
let's build new tools
what shoudl we build?
what problems to address?
ease of testing
``Works On My Machine''
getting a better idea about how a strategy would perform against more realistic collection of peers

dropbox checksubmit
learned how cs dropbox works
a little bash + make to copy in the student handout, detect and write a config file to run their code
show script + makefile
show an output

leaderboard
new tools: cycles cron
simple to copy in all submissions, detect and write config file, run code, post it to website
discuss issues
student code breaking it
harden the tester (try catch around each student with selfless behavior)
student code causing compile failures
no good solution here
add check submit button to leaderboard and ask students to use it

\section*{Results}
As the goals of a project naturally define evaluation criteria, I have evaluated my project on the basis of each those goals: decreasing the work to build each assignment within a semester and the instructor time allocated to assignment implementation, eliminate the inter-year assignment reconstruction, improve available student resources, and in doing so improve student and instructor satisfaction.

\begin{itemize}
\item discover modularity and decrease assignment to assignment rewriting
  \begin{itemize}
  \item code rewriting measurements
  \item instructor time usage
  \end{itemize}
\item improve student resources
  \begin{itemize}
  \item list resources created
  \item polling data about student usage
  \item empirical results about student error rates
  \end{itemize}
\item create institutional memory and decrease year-to-year rewriting
  \begin{itemize}
  \item cant measure until next year, but set up for it
  \end{itemize}
\item increase student and instructor satisfaction
  \begin{itemize}
  \item polling data about student usage
  \end{itemize}
\end{itemize}

%Thus, quantitative metrics, including software failure rates and instructor time usage, and qualitiative metrics, including student and instructor satisfaction, both illustrate the improvements to the course and emphasize the importance of quality tooling to quality programming-based assignments.

\section*{Conclusions}
Through this work, I aim to improve the quality of computer science theory education, from the perspectives of both the students and faculty associated with COS 445.

prisoner's dilemma + search auctions

%The goal of this project is to discover to what degree the development of new strategy design exercises can be simplified or automated, to improve the student resources available during the completion of these exercises, and to create institutional memory and decrease year-to-year repetition of the same work.

%the "programming assignments" aren't really about your ability to code. They are about placing the theoretical results from the course in context. They create a scenario in which mastery of theoretical results from the course content is demonstrated by making and justifying real-world decisions.

\section*{Future Work}

Some of the implementations of my infrastructure could be simplified using ClassLoaders, which would replace some of the Python code.
This improvement was inspired by the implementation of ClassLoaders in the Spring 2014 student handout code.

would like to not use separate files even if lots of the code is the same - complicates handout but simplifies code reuse and iteration

Earlier assignments were built off of earlier versions of the infrastructure and will need to be ported to newer versions when they are reused: do next year

will be reused in two years without my oversight

other universities? matt through academic connections

\section*{Acknowledgements}
* matt for assignment design lead
* course staff, especially Andreea, Ariel, Evan, Heesu, Leila, Divya, Matheus, Wei, Dylan, for grading student writeups, providing me feedback and passing along student feedback
* david walker for iw sem (helping me stay organized and on track, helping me identify my topic)
* iw sem peers Greg, Alex, Mel, Noah for feedback
* jeremie for creating github repo
* kate for providing writing help and moral support

\section*{Ethics}
* assignments used in the course were used by the executive decision of Professor Weinberg,
* potential conflicts of interest
* honor code

\end{document}

