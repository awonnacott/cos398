\documentclass[pageno]{jpaper}

\newcommand{\Semester}{Spring}
\newcommand{\IWreport}{2018}
\newcommand{\quotes}[1]{``#1''}

\widowpenalty=9999

\usepackage[normalem]{ulem}

\begin{document}

\title{
A Modular Framework for Developing, Deploying, and Evaluating Game-Theoretic Strategy Design Exercises}

\author{Andrew Wonnacott\\
  Advisers: Professor Dave Walker, Professor Matt Weinberg}

\date{}
\maketitle

\thispagestyle{empty}
\doublespacing{}
\begin{abstract}
  Student interest in Princeton's undergraduate course in algorithmic game theory, ``Economics and Computing'' (COS 445), has grown rapidly in recent years.
  This course employs small simulations to develop students' ability to use theoretical tools to analyze more real-world situations.
  These strategy design exercises require considerably more software infrastructure than would a usual theoretical problem set, including handout code for student self-evaluation and software autograders.
  While some frameworks for automated grading exist, none surveyed provided the specific features desired for repeatedly developing very similarly structured assignments and evaluating student submissions in the context of each other.
  In order to facilitate more automated grading of these assignments in this and future semesters, I have developed a framework of modular, extensible tools for building all of the necessary compoenents oif these strategy design assignments.
  Concurrently, in my role as a grader for COS 445, I used the framework to develop and evaluate four strategy design assignments, using assignment specifications developed with Professor Weinberg.
  This provides feedback from real student and instructor users which was used to improve the design.
  Empirical results demonstrate the importance of quality tooling to effective use of instructor time, low broken submission rates, and high student satisfaction.
\end{abstract}

\newpage
\section*{Introduction}
\begin{itemize}
\item Discussion of strategy design assignments on cos445sp1[78] with specific examples (from this year + PD)
\item Mention how they are graded (not zero-sum)
\item While these assignments were popular with students\footnote{https://reg-captiva.princeton.edu/chart/index.php?terminfo=1184\&courseinfo=012095 requires CAS login}, they also demanded considerable instructor resources be dedicated to building and grading software resoruces. Additionally, due to the limited tooling provided for students to evaluate their submissions, many submissions did not compile or caused runtime exceptions, requiring considerable additional instructor resources.
\end{itemize}

\section*{Related Work}
This work was developed to replace previous software design exercises assigned to COS 445 students\footnote{including the author} in Spring 2017.\footnote{https://web.archive.org/web/20170916042726/https://www.cs.princeton.edu/~smattw/Teaching/cos445sp17.htm}
In one problem set, students developed and implemented strategies for three different strategy design exercises, related to three different classical games: Prisoner's Dilemma, Centipede, and Ultimatum.
In a later problem set, an exercise placed students in the role of bidders in a generalized second-price auction.\footnote{an auction of multiple items, one after another, where the winning bid pays the second highest bid. This is the auction format used by Google to sell sponsored search auctions.}

Both problem sets were implemented and graded by Cyril Zhang\footnote{https://github.com/cyrilzhang}, a graduate student TA for the course.
Cyril developed interfaces, dummy implementations, and grader software for each strategy design exercise.\footnote{https://github.com/PrincetonUniversity/cos445-hw/tree/s17 requires invite: contact author or department}
However, this existing implementation of grader software for strategy design assignments is not modular and has to be completely rewritten for each assignment, and was designed for grader use only.
The student interfaces were documented, but the grader software was not documented for future re-use.
Additionally, some of the scripts and notebooks used by Cyril were not preserved and were not accessible as resources.
Thus, while these resources served as inspiration for this work and provided some assignment-specific code to my work developing specific assignments, I did not build directly on Cyril's code.

Previous to Spring 2017, COS 445 was taught by Professor Mark Braverman in Fall 2012 and Spring 2014\footnote{https://precourser.io/course/1184012095 requires Princeton CAS authentication}.
Though no documentation or archives of the course homepage from these semesters is available online or in the Internet Archive, I recovered some resources from the course account on the ``Cycles'' cluster of the Princeton Computer Science department and from the source code respositories used by the course staff.
In both semesters, Yonatan Naamad\footnote{https://github.com/ynaamad/} was responsible for software development.

The only resources available from Fall 2012 were collected from the course Google Code repository.\footnote{https://code.google.com/archive/p/cos445-scripts/}
This repository contains incomplete Python skeleton code for a basic tournament.
This was uploaded to the current course repository,\footnote{https://github.com/PrincetonUniversity/cos445-hw/tree/f12 requires invite: contact author or department} but was not reused.

The Spring 2014 iteration of the course was developed on a private GitHub repository.\footnote{https://github.com/ynaamad/cos445}
Through a clone of this repository stored on Cycles, I discovered that the course had PHP and Java implementations of strategy design exercises for Prisoner's Dilemma and another auction setting.
The Java code consisted of student handouts of interfaces and some testing code, but the testing code was not designed to be automated for grading and was not modular: it tested manually-enterred strategies against each other and assignment-specific details (e.g.\ game structure and payoffs) were integrated within the simulation code.
As a design decision I dismissed the available PHP code, which was used for course grading, due to language unfamiliarity.
These resources were uploaded to the current course repository,\footnote{https://github.com/PrincetonUniversity/cos445-hw/tree/s14 requires invite: contact author or department} but were not reused, in part because they were located and examined only when I was given access to the course site in order to deploy my project.

Beyond COS 445 at Princeton, many universities have taught computer science courses in algorithmic game theory.
However, a survey of several universities' courses, including courses at Harvard\footnote{https://beta.blogs.harvard.edu/k108875/}, Stanford\footnote{https://theory.stanford.edu/~tim/364a.html and https://theory.stanford.edu/~tim/f16/f16.html}, Cornell\footnote{https://www.cs.cornell.edu/courses/cs6840/2014sp/}, U. Washington\footnote{https://courses.cs.washington.edu/courses/cse522/05sp/}, Yale\footnote{https://zoo.cs.yale.edu/classes/cs455/fall11/}, U. Penn\footnote{https://www.cis.upenn.edu/~aaroth/courses/agtS15.html}, Georgia Tech\footnote{https://web.archive.org/web/20131002121909/https://www.cc.gatech.edu/\%7Eninamf/LGO10/index.html}, Carnegie Mellon\footnote{https://www.cs.cmu.edu/~arielpro/15896s16/index.html}, and Duke\footnote{https://www2.cs.duke.edu/courses/spring16/compsci590.4/} identified only one other course with similarly structured assignments.

Harvard's Computer Science 136, ``Economics and Computation'',\footnote{https://beta.blogs.harvard.edu/k108875/assignments/} assigns two programming problem sets.
The first requires implementation of a peer-to-peer filesharing system similar to Bittorrent, a protocol in which incentives parallel Prisoner's Dilemma.
However, the assignment is not a competition and student solutions are not differentiated based on strategic properties.
The second assignment is a strategy design exercise built on the generalized second-price auction, in which student strategies vary and are graded based on their performance in a competitive environment.
This is very similar to both the Spring 2014 and Spring 2017 auction exercises of COS 445 at Princeton.
However, the grading structure of this assignment is quite different from the grading structure of the COS 445 strategy design exercises, and the implementation is in Python.
While comparison between COS 445 assignments and CS 136 assignments in a future project might provide some pedagogical value, such work would be out of scope for this project.
Given that these assignments are implemented in Python, only handout (and not grading) code is available, and the assignments are graded differently, these resources were dismissed as a base for development of new resources for COS 445 at Princeton.

Many universities, including Princeton, maintain extensive systems for the automatic submission and processing of student assignments.
Such resources are useful in the general case of developing new programming-based assignments, but no such resources replicate the behavior implemented in this project.
Most autograding systems are designed with an assumption that student submission are graded in a sandbox, isolated from other student submissions.
Such systems are not applicable to this project, since the strategy design exercises involve evaluation of student strategy performance in a common environment.
However, existing resources for assignment submission and student feedback in use by Princeton are used in this assignment.
I integrated some CS DropBox\footnote{https://csguide.cs.princeton.edu/academic/csdropbox} functionality where relevant, and modelled the leaderboard structure after review of the leaderboard implementation used in other Princeton CS classes.
Generally, however, no such tools would leverage the high degree of structural similarity between strategy design exercises, and a dedicated project for these exercises will allow for more efficient development and evaluation.

\section*{Conclusions}
prisoner's dilemma + search auctions

\section*{Future Work}

Some of the implementations of my infrastructure could be simplified using ClassLoaders, which would replace some of the Python code.
This improvement was inspired by the implementation of ClassLoaders in the Spring 2014 student handout code.

Earlier assignments were built off of earlier versions of the infrastructure and will need to be ported to newer versions when they are reused

\end{document}

