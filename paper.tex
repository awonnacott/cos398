\documentclass[pageno]{jpaper}

\newcommand{\Semester}{Spring}
\newcommand{\IWreport}{2018}
\newcommand{\quotes}[1]{``#1''}

\widowpenalty=9999

\usepackage[normalem]{ulem}

\begin{document}

\title{
A Modular Framework for Developing, Deploying, and Evaluating Game-Theoretic Strategy Design Exercises}

\author{Andrew Wonnacott\\
  Advisers: Professor Dave Walker, Professor Matt Weinberg}

\date{}
\maketitle

\thispagestyle{empty}
\doublespacing{}
\begin{abstract}
  Student interest in Princeton's undergraduate course in algorithmic game theory, ``Economics and Computing'' (COS 445), has grown rapidly in recent years.
  This course employs small simulations to develop students' ability to use theoretical tools to analyze more real-world situations.
  These strategy design exercises require considerably more software infrastructure than would a usual theoretical problem set, including handout code for student self-evaluation and software autograders.
  While some frameworks for automated grading exist, none surveyed provided the specific features desired for repeatedly developing very similarly structured assignments and evaluating student submissions in the context of each other.
  In order to facilitate more automated grading of these assignments in this and future semesters, I have developed a framework of modular, extensible tools for building all of the necessary compoenents oif these strategy design assignments.
  Concurrently, in my role as a grader for COS 445, I used the framework to develop and evaluate four strategy design assignments, using assignment specifications developed with Professor Weinberg.
  This provides feedback from real student and instructor users which was used to improve the design.
  Empirical results demonstrate the importance of quality tooling to effective use of instructor time, low broken submission rates, and high student satisfaction.
\end{abstract}

\newpage
\section*{Introduction}
Four Princeton University departments offer courses in game theory: economics, politics, mathematics, and computer science.
While all of these courses teach the core concepts of game theory, each also connects this core to relevant examples and topics in its department.
For instance, in the computer science department, COS 445, ``Economics and Computing'', places emphasis on the connections between game-theoretic results and algorithms, software, and other aspects of computer science.
This is represented in the course topics, which focus on the relevance of game theory to matchings, auctions, cryptocurrencies, and other multi-actor computational systems.\footnote{https://www.cs.princeton.edu/~smattw/Teaching/cos445sp18.htm}
It is also represented in the design of the problem sets, which contain, in addition to traditional proof-based exercises, programming-based exercises which connect course topics to prerequesite computer science skills.
These programming-based exercises take a common form.

Students are presented with an iterated game, described both in the assignment handout and as a Java interface.
A classic iterated game is the iterated prisoner's dilemma\footnote{Figure: PD game -> Java iterface}, in which two players must collaborate to receive the best outcome, but are each incentivized to defect, which results in a less preferred outcome, and have acecss to the other player's play history when making their own decision.
While this game is well-studied and models some real-world incentive structures, the course also draws on various real-world phenomenon to design iterated games, including, from semester to semester, a college admissions game from the perspective of stable matchings, a search advertisement bidding game using auction theory, an election informed by voting theory, a gerrymandering-prevention simulation inspired by fair division, and exploitation of a fictional cryptocurrency by analyzing blockchain incentives.
These examples are not exhaustive, but illustrate the variety of course topics with potential for translation to iterated games.

Students then craft strategies to play the specified game, based on their own analyses of the game's equilbirium, and based on their own priors over the strategies which will be designed by their peers.
These strategies are evaluated on both on their design quality and on their empirical performance.
While design quality is evaluated on a purely individual basis, empirical performance is measured by simulating the play of the collection of submitted student strategies.
For instance, for the iterated prisoner's dilemma, every student strategy plays one full game (consisting of many iterations of the prisoner's dilemma) against each other strategy, and strategies are scored based off the average payoff they achieved.
For the search advertisement game, all the student strategies are concurrently asked to bid on the same search terms, and are scored based on their revenue from ad clicks (where each bidder knows the click-through rate for a term prior to bidding on that term) less their expenditures on advertising.
To create a more realistic simulation, students are graded solely the performance of their own strategy, not on a curve of the performance of their peers.
That is, student grades on these strategy design exercises are not zero-sum.

While the strategy design exercises assignments are popular with students\footnote{https://reg-captiva.princeton.edu/chart/index.php?terminfo=1184\&courseinfo=012095 requires CAS login}, they also demanded considerable instructor resources be dedicated to building and grading software resoruces, which have historically been significantly rebuilt for each assignment.
Worse still, due to course staffing changes, this work has not been preserved between semesters, resulting in inefficient allocation of department resources to producing new exercises without the institutional memory or software resources previously developed.
Due to the limited tooling provided in past semesters for students to evaluate their submissions, many submissions did not compile or caused runtime exceptions, requiring considerable additional instructor resources.

The goal of this project is to discover to what degree the development of new strategy design exercises can be simplified or automated, to improve the student resources available during the completion of these exercises, and to create institutional memory and decrease year-to-year repetition of the same work.
This was accomplished by designing a reuasble framework which embodies the commonality between strategy design exercises, allowing instructors to change only the necessary, assignment-specific details, and which will include more advanced automated grading tools than previously existed for the course.
Within this framework, I have built a collection of utilities to facilitate student self-evaluation and decrease strategy iteration overhead.

Through this work, I aim to improve the quality of computer science theory education, from the perspectives of both the students and faculty associated with COS 445.
Quantitative metrics, including software failure rates and instructor time usage, and qualitiative metrics, including student and instructor satisfaction, both illustrate the improvements to the course and emphasize the importance of quality tooling to quality programming-based assignments.

\section*{Related Work}
This work was developed to replace previous software design exercises assigned to COS 445 students\footnote{including the author} in Spring 2017.\footnote{https://web.archive.org/web/20170916042726/https://www.cs.princeton.edu/~smattw/Teaching/cos445sp17.htm}
In one problem set, students developed and implemented strategies for three different strategy design exercises, related to three different classical games: Prisoner's Dilemma, Centipede, and Ultimatum.
In a later problem set, an exercise placed students in the role of bidders in a generalized second-price auction.\footnote{an auction of multiple items, one after another, where the winning bid pays the second highest bid. This is the auction format used by Google to sell sponsored search auctions.}

Both problem sets were implemented and graded by Cyril Zhang\footnote{https://github.com/cyrilzhang}, a graduate student TA for the course.
Cyril developed interfaces, dummy implementations, and grader software for each strategy design exercise.\footnote{https://github.com/PrincetonUniversity/cos445-hw/tree/s17 requires invite: contact author or department}
However, this existing implementation of grader software for strategy design assignments is not modular and has to be completely rewritten for each assignment, and was designed for grader use only.
The student interfaces were documented, but the grader software was not documented for future re-use.
Additionally, some of the scripts and notebooks used by Cyril were not preserved and were not accessible as resources.
Thus, while these resources served as inspiration for this work and provided some assignment-specific code to my work developing specific assignments, I did not build directly on Cyril's code.

Previous to Spring 2017, COS 445 was taught by Professor Mark Braverman in Fall 2012 and Spring 2014\footnote{https://precourser.io/course/1184012095 requires Princeton CAS authentication}.
Though no documentation or archives of the course homepage from these semesters is available online or in the Internet Archive, I recovered some resources from the course account on the ``Cycles'' cluster of the Princeton Computer Science department and from the source code respositories used by the course staff.
In both semesters, Yonatan Naamad\footnote{https://github.com/ynaamad/} was responsible for software development.

The only resources available from Fall 2012 were collected from the course Google Code repository.\footnote{https://code.google.com/archive/p/cos445-scripts/}
This repository contains incomplete Python skeleton code for a basic tournament.
This was uploaded to the current course repository,\footnote{https://github.com/PrincetonUniversity/cos445-hw/tree/f12 requires invite: contact author or department} but was not reused.

The Spring 2014 iteration of the course was developed on a private GitHub repository.\footnote{https://github.com/ynaamad/cos445}
Through a clone of this repository stored on Cycles, I discovered that the course had PHP and Java implementations of strategy design exercises for Prisoner's Dilemma and another auction setting.
The Java code consisted of student handouts of interfaces and some testing code, but the testing code was not designed to be automated for grading and was not modular: it tested manually-enterred strategies against each other and assignment-specific details (e.g.\ game structure and payoffs) were integrated within the simulation code.
As a design decision I dismissed the available PHP code, which was used for course grading, due to language unfamiliarity.
These resources were uploaded to the current course repository,\footnote{https://github.com/PrincetonUniversity/cos445-hw/tree/s14 requires invite: contact author or department} but were not reused, in part because they were located and examined only when I was given access to the course site in order to deploy my project.

Beyond COS 445 at Princeton, many universities have taught computer science courses in algorithmic game theory.
However, a survey of several universities' courses, including courses at Harvard\footnote{https://beta.blogs.harvard.edu/k108875/}, Stanford\footnote{https://theory.stanford.edu/~tim/364a.html and https://theory.stanford.edu/~tim/f16/f16.html}, Cornell\footnote{https://www.cs.cornell.edu/courses/cs6840/2014sp/}, U. Washington\footnote{https://courses.cs.washington.edu/courses/cse522/05sp/}, Yale\footnote{https://zoo.cs.yale.edu/classes/cs455/fall11/}, U. Penn\footnote{https://www.cis.upenn.edu/~aaroth/courses/agtS15.html}, Georgia Tech\footnote{https://web.archive.org/web/20131002121909/https://www.cc.gatech.edu/\%7Eninamf/LGO10/index.html}, Carnegie Mellon\footnote{https://www.cs.cmu.edu/~arielpro/15896s16/index.html}, and Duke\footnote{https://www2.cs.duke.edu/courses/spring16/compsci590.4/} identified only one other course with similarly structured assignments.

Harvard's Computer Science 136, ``Economics and Computation'',\footnote{https://beta.blogs.harvard.edu/k108875/assignments/} assigns two programming problem sets.
The first requires implementation of a peer-to-peer filesharing system similar to Bittorrent, a protocol in which incentives parallel Prisoner's Dilemma.
However, the assignment is not a competition and student solutions are not differentiated based on strategic properties.
The second assignment is a strategy design exercise built on the generalized second-price auction, in which student strategies vary and are graded based on their performance in a competitive environment.
This is very similar to both the Spring 2014 and Spring 2017 auction exercises of COS 445 at Princeton.
However, the grading structure of this assignment is quite different from the grading structure of the COS 445 strategy design exercises, and the implementation is in Python.
While comparison between COS 445 assignments and CS 136 assignments in a future project might provide some pedagogical value, such work would be out of scope for this project.
Given that these assignments are implemented in Python, only handout (and not grading) code is available, and the assignments are graded differently, these resources were dismissed as a base for development of new resources for COS 445 at Princeton.

Many universities, including Princeton, maintain extensive systems for the automatic submission and processing of student assignments.
Such resources are useful in the general case of developing new programming-based assignments, but no such resources replicate the behavior implemented in this project.
Most autograding systems are designed with an assumption that student submission are graded in a sandbox, isolated from other student submissions.
Such systems are not applicable to this project, since the strategy design exercises involve evaluation of student strategy performance in a common environment.
However, existing resources for assignment submission and student feedback in use by Princeton are used in this assignment.
I integrated some CS DropBox\footnote{https://csguide.cs.princeton.edu/academic/csdropbox} functionality where relevant, and modelled the leaderboard structure after review of the leaderboard implementation used in other Princeton CS classes.
Generally, however, no such tools would leverage the high degree of structural similarity between strategy design exercises, and a dedicated project for these exercises will allow for more efficient development and evaluation.

\section*{Approach}
Throughout the course of developing this framework, I have leveraged my role as an undergraduate course assistant for COS 445 during the current semester.


That is, in order to develop a reusable toolchain for building strategy design exercises, I have developed strategy design exercises to inform the design and development of a reusable toolchain.

* how to discover what can be modularized

\section*{Implementation}
* how to modularize that

\section*{Results}

\section*{Conclusions}
prisoner's dilemma + search auctions

\section*{Future Work}

Some of the implementations of my infrastructure could be simplified using ClassLoaders, which would replace some of the Python code.
This improvement was inspired by the implementation of ClassLoaders in the Spring 2014 student handout code.

Earlier assignments were built off of earlier versions of the infrastructure and will need to be ported to newer versions when they are reused: do next year

will be reused in two years without my oversight

other universities? matt through academic connections

\section*{Acknowledgements}

\section*{Ethics}
* assignments used in the course were used by the executive decision of Professor Weinberg,
* potential conflicts of interest
* honor code

\end{document}

